\documentclass[12pt, a4paper]{article}
\usepackage[utf8]{inputenc}
\usepackage[T1]{fontenc}
\usepackage[slovene]{babel}
\usepackage{lmodern}
\usepackage{amsmath}
\usepackage{units}
\usepackage{eurosym}
\usepackage{amsfonts}

\usepackage{enumerate}
\setlength{\parindent}{0mm}

\newcommand{\novukaz}[2]{\underline {#1} \textit{#2}}

\newcounter{stevec}

\newenvironment{novookolje}[2]{\stepcounter{stevec} #1 #2 \thestevec}{}


\begin{document}
\begin{titlepage}
\begin{center}

\large
Univerza v Ljubljani\\
\normalsize
Fakulteta za matematiko in fiziko\\

\vspace{3 cm} 

\large
Ines Šilc in Eva Deželak \\

\vspace{0.5cm}
\LARGE
\textbf{KOCKARJEV PROPAD}

\vspace{0.5 cm}
\normalsize
Seminar

\vspace{1.5cm}
\normalsize
Mentor: doc. dr. Matija Vidmar

\vspace{3cm}


\vfill

\large Ljubljana, 2019

\end{center}
\end{titlepage}

\newpage

\tableofcontents

\newpage

 \section[Besedilo naloge]{Besedilo naloge}


\begin{flushleft}
Igralec $M$ ima $m$ enot denarja, igralec $N$ pa $n$ enot premoženja, $\{m, n\}$ $\subset$ $\mathbb{N}$. Zapored igrata igro na srečo v kateri ni neodločenih izidov; v vsaki igri dobi zmagovalec eno denarno enoto od poraženca; igralec $M$ zmaga vsakič z verjetnostjo $p \in (0, 1)$, neodvisno od preteklosti. Igranje se konča, ko eden od igralcev bankrotira. Naj bo $T$ število iger, ki je potrebnih, da eden od igralcev bankrotira.

\begin{itemize}
    \item Predpostavi, da je $T < \inf$ s.g. (*ali znaš to utemeljiti?). Določi verjetnost, da bankrotira igralec $M$.
    \item Predpostavi, da je $E[T] < \inf$. Določi $E[T]$.
\end{itemize}
\end{flushleft}

\end{document}