\documentclass[12pt, a4paper]{article}
\usepackage[utf8]{inputenc}
\usepackage[T1]{fontenc}
\usepackage[slovene]{babel}
\usepackage{lmodern}
\usepackage{amsmath}
\usepackage{eurosym}
\usepackage{amsfonts}

\usepackage{enumerate}
\setlength{\parindent}{0mm}

\newcommand{\novukaz}[2]{\underline {#1} \textit{#2}}

\newcounter{stevec}

\newenvironment{novookolje}[2]{\stepcounter{stevec} #1 #2 \thestevec}{}


\begin{document}
\begin{titlepage}
\begin{center}

\large
Univerza v Ljubljani\\
\normalsize
Fakulteta za matematiko in fiziko\\

\vspace{3 cm} 

\large
Eva Deželak in Ines Šilc\\

\vspace{0.5cm}
\LARGE
\textbf{KOCKARJEV PROPAD}

\vspace{0.5 cm}
\normalsize
Seminar

\vspace{1.5cm}
\normalsize
Mentor: doc. dr. Matija Vidmar

\vspace{3cm}


\vfill

\large Ljubljana, 2019

\end{center}
\end{titlepage}

\newpage

\tableofcontents
\vspace{20mm}

 \section[Besedilo naloge]{Besedilo naloge}



Igralec $M$ ima $m$ enot denarja, igralec $N$ pa $n$ enot premoženja, $\{m, n\}$ $\subset$ $\mathbb{N}$. Zapored igrata igro na srečo v kateri ni neodločenih izidov; v vsaki igri dobi zmagovalec eno denarno enoto od poraženca; igralec $M$ zmaga vsakič z verjetnostjo $p \in (0, 1)$, neodvisno od preteklosti. Igranje se konča, ko eden od igralcev bankrotira. Naj bo $T$ število iger, ki je potrebnih, da eden od igralcev bankrotira.

\begin{itemize}
    \item Predpostavi, da je $T < \infty$ s.g. (*ali znaš to utemeljiti?). Določi verjetnost, da bankrotira igralec $M$.
    \item Predpostavi, da je $E[T] < \infty$. Določi $E[T]$.
\end{itemize}


\newpage

\section[Rešitev]{Rešitev}
\subsection{Računanje verjetnosti, da bankrotira igralec M}

Če je verjetnost, da zmaga igralec $M$ enaka $p$, potem je verjetnost, da zmaga igralec $N$ enaka $1-p$. Naj bo $A_a$ dogodek, da bankrotira igralec M, če ima trenutno a denarja. Za vsako naravno ptevilo a definiramo $p_a = P(A_a)$.
Poiskati želimo $p_m$, torej verjetnost, da propade igralec $M$, če ima $m$ denarja. 
\\

Naši hipotezi sta:
\\
$H$ - prvi igralec v naslednjem krogu zmaga (torej dobi 1 od drugega igralca)
\\
$H^C$ - prvi igralec v naslednjem krogu izgubi (torej da 1 drugemu igralcu)
\\

Verjetnost, da igralec $M$ propade, če ima $a$ denarja je enaka:
\\
$$P(A_a) = P(A_a / H) \cdot P(H) + P(A_a / H^C) \cdot P(H^C)$$ iz česar sledi
$$p_a = P_{a+1} \cdot p + p_{a-1} \cdot (1 - p) $$
$$(p + 1 -p) \cdot p_a = P_{a+1} \cdot p + p_{a-1} \cdot (1 - p) $$
$$ p \cdot p_a + (1 - p) \cdot p_a = p_{a+1} \cdot p + p_{a-1} \cdot (1 - p) $$
$$p \cdot p_a - P_{a+1} \cdot p = p_{a-1} - (1-p) \cdot p_a$$
$$p \cdot (p_a - p_{a+1}) = (1-p) \cdot (p_{a-1} - p_a)$$
\\

Sedaj uvedemo novo oznako: $u_a = p_a - p_{a+1}$. Torej velja tudi $u_{a-1} = p_{a-1} - p_a$, zato sledi
$$p \cdot u_a = (1 - p) \cdot u_{a-1}$$
Torej $$u_a = \frac{1-p}{p} \cdot u_{a-1}$$

Če sedaj uvedemo še oznako $r=\frac{1-p}{p}$, dobimo
$$u_a = r \cdot u_{a-1}$$

Od tod lahko izrazimo vse člene z začetnim (torej z $u_0$):
$$u_1 = r \cdot u_0$$
$$u_2 = r \cdot u_1 = r \cdot r \cdot u_0 = r^2 \cdot u_0$$ 
$$\cdots $$
$$u_a = r^a \cdot u_0$$
\newpage

Vemo, da če ima eden od igralcev ves denar, potem gotovo zmaga, če pa ga nima nič, potem gotovo izgubi, zato je:
$$p_c = 0 \textrm{, kjer je } c = m + n \textrm{ in } p_0 = 1$$

Sedaj lahko najprej $u_0$ izrazimo z začetnimi podatki:
\begin{equation*}
\begin{split}
1 = p_0 - p_c &= (p_0 - p_1) + (p_1 - p_2) + \cdots +               (p_{c-1} - p_c) = \\
        &= u_0 + u_1 + u_2 + \cdots + u_{c-1} = \\
        &= u_0 + r \cdot u_0 + r^2 \cdot u_0 + \cdots + r^{c-1} \cdot u_0 = \\
        &= u_0 \cdot (1 + r + r^2 + \cdots + r^{c-1}) = \\
        &= u_o \cdot \frac{r^c - 1}{r-1}
\end{split}
\end{equation*} 

Iz tega sledi $$u_0 = \frac{r-1}{r^c -1}$$

Sedaj lahko izračunamo verjetnost za propad prvega igralca (vemo, da je $u_m = p_m - p_{m+1}$):
\begin{equation*}
\begin{split}
p_m &= u_m + p_{m+1} = \\
    &= u_m + u_{m+1} + p_{m+2} = \\
    &= u_m + u_{m+1} + u_{m+2} + p_{m+3} = \\
    &= u_m + u_{m+1} + u_{m+2} + u_{m+3} + \cdots + u_{m+n-1} + p_{m+n} = \\
    &= u_m + u_m \cdot r + u_m \cdot r^2 + \cdots + u_m \cdot r^{n-1} + 0 = \\
    &= u_m + u_m \cdot \frac{1-p}{p} + u_m \cdot (\frac{1-p}{p})^2 + u_m \cdot (\frac{1-p}{p})^3 + \cdots + u_m \cdot (\frac{1-p}{p})^{n-1} = \\
    &= u_m \cdot (1 + \frac{1-p}{p} + (\frac{1-p}{p})^2 + \cdots + (\frac{1-p}{p})^{n-1}) 
    = u_m \cdot \frac{(\frac{1-p}{p})^n - 1}{\frac{1-p}{p} - 1} = \\
    &= r^m \cdot u_0 \cdot \frac{(\frac{1-p}{p})^n - 1}{\frac{1-p}{p} - 1}  
    = (\frac{1-p}{p})^m \cdot \frac{\frac{1-p}{p} - 1}{(\frac{1-p}{p})^{m+n} -1}\cdot \frac{(\frac{1-p}{p})^n - 1}{\frac{1-p}{p} - 1} = \\
    &= \frac{(\frac{1-p}{p})^n - 1}{(\frac{1-p}{p})^{m+n} -1} \cdot (\frac{1-p}{p})^m 
    = \frac{(\frac{1-p}{p})^{m+n} - (\frac{1-p}{p})^m}{(\frac{1-p}{p})^{m+n} -1} = \\
    &= \frac{\frac{(1-p)^{n+m}}{p^n} - (1-p)^m}{\frac{(1-p)^{n+m}}{p^n} - p^m} = 
    \frac{(1-p)^{n+m} - (1-p)^m \cdot p^n}{(1-p)^{m+n} - p^{m+n}}
\end{split} 
\end{equation*}
Torej dobili smo verjetnost za propad igralca $M$.


\subsection{Računanje $E[T]$}

Rešujemo enačbo oblike $E[T|M=x] = p\cdot E[T|M=x+1] + (1-p)\cdot E[T|M=x-1] + 1$, kar zaradi lažjega zapisa prevedemo na obliko $f(x) = p\cdot f(x+1) + (1-p) \cdot f(x-1) + 1 ~ \textrm{oziroma} $ $$p\cdot f(x+2) - f(x+1) + (1-p)\cdot f(x) = -1$$ kjer je $f(x)=E[T|M=x]$. Rešujemo torej nehomogeno rekurzivno enačbo, katero rešitev je vsota homogene in partikularne rešitve. 
\\
Rešimo najprej homogeni del. Rešujemo enačbo: $$p\cdot f(x+2) - f(x+1) + (1-p)\cdot f(x) = 0$$ s pomočjo karakterističnega polinoma dobimo enačbo: $$p\cdot  \lambda ^2 - \lambda + (1-p) = 0$$ $$\lambda _{1, 2}= \frac{1 \pm \sqrt{1 - 4p(1-p)}}{2p} = \frac{1 \pm (2p-1)}{2p}.$$
Dobimo $\lambda _{1} = 1$ in $\lambda _{2} = \frac{1-p}{p}$. Rešitev homogene enačbe je torej oblike $A\cdot 1^x + B(\frac{1-p}{p})^x$, oziroma: $$A+ B \bigg( \frac{1-p}{p} \bigg )^x$$

Ker imamo homogeno rešitev sestavljeno iz dveh delov, bomo partikularni del prav tako poiskali v dveh delih. Prvi del iščemo z nastavkom $f(x)= Cx1^x$ oziroma $f(x)= Cx$. Vstavimo v prvotno enačbo in dobimo: $$pC(x+1)-C(x+1)+(1-p)Cx=-1$$ $$Cpx+2Cp-Cx-C+Cx-Cpx=-1$$ $$C(2p-1)=-1$$ $$C=\frac{-1}{2p-1} = \frac{1}{1-2p}$$ Prvi del partikularne enačbe je: $$f(x)=\frac{x}{1-2p}$$

Drugi del iščemo z nastavkom $f(x) = D x (\frac{1-p}{p})^x$. Vstavimo v prvotno enačbo in dobimo:
$$pD(x+2)\bigg( \frac{1-p}{p} \bigg )^{(x+2)} - D(x+1)\bigg( \frac{1-p}{p} \bigg )^{(x+1 )}+(1-p)Dx\bigg( \frac{1-p}{p} \bigg )^x = -1$$ 
$$D \cdot \bigg( \frac{1-p}{p} \bigg )^x \cdot \bigg( px\bigg( \frac{1-p}{p} \bigg )^2 + 2p\bigg( \frac{1-p}{p} \bigg )^2 -x\bigg( \frac{1-p}{p} \bigg )-\bigg( \frac{1-p}{p} \bigg )+x-px \bigg )= -1$$ 
$$D = \frac{-1}{\big( \frac{1-p}{p} \big )^x \cdot \big( px\big( \frac{1-p}{p} \big )^2 + 2p\big( \frac{1-p}{p} \big )^2 -x\big( \frac{1-p}{p} \big )-\big( \frac{1-p}{p} \big )+x-px \big )}$$ 
\\
Drugi del partikularne enačbe je: $$f(x) = \frac{-x  (\frac{1-p}{p})^x}{\big( \frac{1-p}{p} \big )^x \cdot \big( px\big( \frac{1-p}{p} \big )^2 + 2p\big( \frac{1-p}{p} \big )^2 -x\big( \frac{1-p}{p} \big )-\big( \frac{1-p}{p} \big )+x-px \big )}$$ $$= \frac{x}{(x+1)\big( \frac{1-p}{p} \big )-\big( \frac{1-p}{p} \big )^2(px+2p)+px-1},$$\\
celotna partikularna rešitev pa je vsota homogene in partikularne: $$f(x)=\frac{x}{1-2p}+\frac{x}{(x+1)\big( \frac{1-p}{p} \big )-\big( \frac{1-p}{p} \big )^2(px+2p)+px-1}.$$\\

Splošna rešitev prvotne enačbe je: $$A + B\bigg( \frac{1-p}{p} \bigg )^x+\frac{x}{1-2p}+\frac{x}{(x+1)\big( \frac{1-p}{p} \big )-\big( \frac{1-p}{p} \big )^2(px+2p)+px-1}$$

Upoštevamo lahko še robna pogoja:
\begin{enumerate}
\item Pričakovano število iger, če smo brez denarja je 0, saj se takrat igra konča. Pogoj zapišemo v obliki $E[T|M= 0] = f(0) = 0$. Vstavimo v splošno rešitev in dobimo:
$$A + B\bigg( \frac{1-p}{p} \bigg )^0 = 0, \quad A = -B$$
\item Pričakovano število iger, če imamo m+n denarja je 0, saj se takrat igra konča. Pogoj zapišemo v obliki $E[T|M= m+n] = f(m+n) = 0$. Vstavimo v splošno rešitev in dobimo:\\\\
$-B + B\bigg( \frac{1-p}{p} \bigg )^{m+n}+\frac{m+n}{1-2p}+\frac{m+n}{(m+n+1)\big( \frac{1-p}{p} \big )-\big( \frac{1-p}{p} \big )^2(p(m+n)+2p)+p(m+n)-1}=0$\\

$$B = \frac{\frac{m+n}{1-2p}+\frac{m+n}{(m+n+1)\big( \frac{1-p}{p} \big )-\big( \frac{1-p}{p} \big )^2(p(m+n)+2p)+p(m+n)-1}}{1-\big( \frac{1-p}{p} \big )^{m+n}}$$

$$A = \frac{\frac{m+n}{1-2p}+\frac{m+n}{(m+n+1)\big( \frac{1-p}{p} \big )-\big( \frac{1-p}{p} \big )^2(p(m+n)+2p)+p(m+n)-1}}{\big( \frac{1-p}{p} \big )^{m+n}-1}$$
\end{enumerate}

Pričakovano število iger, če imamo $x$ enot denarja je: $$E[ T | M = x]= \frac{\frac{m+n}{1-2p}+\frac{m+n}{(m+n+1)\big( \frac{1-p}{p} \big )-\big( \frac{1-p}{p} \big )^2(p(m+n)+2p)+p(m+n)-1}}{\big( \frac{1-p}{p} \big )^{m+n}-1}+$$

 $$+\frac{\frac{m+n}{1-2p}+\frac{m+n}{(m+n+1)\big( \frac{1-p}{p} \big )-\big( \frac{1-p}{p} \big )^2(p(m+n)+2p)+p(m+n)-1}}{1-\big( \frac{1-p}{p} \big )^{m+n}}\big( \frac{1-p}{p} \big )^x+$$ $$+\frac{x}{1-2p}+\frac{x}{(x+1)\big( \frac{1-p}{p} \big )-\big( \frac{1-p}{p} \big )^2(px+2p)+px-1}$$
 Zanima nas $E[T|M=m]$, vstavimo $x=m$ v zgornjo enačbo in dobimo, da je pričakovano število iger, če začnemo z $m$ enotami denarja enako: $$E[T|M=m]=\frac{n \left(\left(\frac{1}{p}-1\right)^m-1\right)-m \left(\frac{1}{p}-1\right)^m
   \left(\left(\frac{1}{p}-1\right)^n-1\right)}{(2 p-1)
   \left(\left(\frac{1}{p}-1\right)^{m+n}-1\right)}$$
 \\
 Formula ne drži če je $p = \frac{1}{2}$, saj pride do deljenja z 0, zato moramo to izračunati posebej. Prav tako rešujemo nehomogeno rekurzivno enačbo, ki jo zapišemo kot: $$\frac{1}{2}f(x+2)-f(x+1)+\frac{1}{2}f(x)=-1$$

Rešimo najprej homogeni del. Rešujemo enačbo: $$\frac{1}{2}f(x+2)-f(x+1)+\frac{1}{2}f(x)=0$$ s pomočjo karakterističnega polinoma dobimo enačbo: $$\frac{1}{2}\lambda^2-\lambda+\frac{1}{2}=0$$ Rešitev je dvojna ničla $\lambda_{1, 2}= 1$. Homogena rešitev je oblike: $$(Ax+B)\cdot 1^x=(Ax+B)$$

Partikularno rešitev iščemo z nastavkom $f(x)=C\cdot x^2\cdot 1^x= Cx^2$. Vstavimo v prvotno enačbo in dobimo:
$$C(x+2)^2-2C(x+1)^2+Cx^2=-2$$
$$C(x^2+4x+4)-2C(x^2+2x+1)+Cx^2=-2$$
$$2C=-2, \quad C = -1 $$ Partikularna rešitev je $f(x)=-x^2$, splošna rešitev pa je: $$f(x)=Ax+B-x^2$$
Vstavimo robne pogoje:
\begin{enumerate}
\item $f(0)=0\Rightarrow B = 0$
\item $f(m+n)=0 \Rightarrow A = m+n$
\end{enumerate}
Pričakovano število iger, če imamo $x$ enot denarja in je verjetnost za zmago $p=\frac{1}{2}$ je: $$E[T|M=x]= (m+n)x-x^2$$
Zanima nas pričakovano število iger, če začnemo z $m$ enotami denarja. Odgovor je:$$E[T|M=m]=(m+n)m-m^2= n\cdot m$$


\end{document}