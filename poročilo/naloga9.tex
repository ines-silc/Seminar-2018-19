\documentclass[12pt, a4paper]{article}
\usepackage[utf8]{inputenc}
\usepackage[T1]{fontenc}
\usepackage[slovene]{babel}
\usepackage{lmodern}
\usepackage{amsmath}
\usepackage{eurosym}
\usepackage{amsfonts}

\usepackage{enumerate}
\setlength{\parindent}{0mm}

\newcommand{\novukaz}[2]{\underline {#1} \textit{#2}}

\newcounter{stevec}

\newenvironment{novookolje}[2]{\stepcounter{stevec} #1 #2 \thestevec}{}


\begin{document}
\begin{titlepage}
\begin{center}

\large
Univerza v Ljubljani\\
\normalsize
Fakulteta za matematiko in fiziko\\

\vspace{3 cm} 

\large
Eva Deželak in Ines Šilc\\

\vspace{0.5cm}
\LARGE
\textbf{KOCKARJEV PROPAD}

\vspace{0.5 cm}
\normalsize
Seminar

\vspace{1.5cm}
\normalsize
Mentor: doc. dr. Matija Vidmar

\vspace{3cm}


\vfill

\large Ljubljana, 2019

\end{center}
\end{titlepage}

\newpage

\tableofcontents

\newpage

 \section[Besedilo naloge]{Besedilo naloge}


\begin{flushleft}
Igralec $M$ ima $m$ enot denarja, igralec $N$ pa $n$ enot premoženja, $\{m, n\}$ $\subset$ $\mathbb{N}$. Zapored igrata igro na srečo v kateri ni neodločenih izidov; v vsaki igri dobi zmagovalec eno denarno enoto od poraženca; igralec $M$ zmaga vsakič z verjetnostjo $p \in (0, 1)$, neodvisno od preteklosti. Igranje se konča, ko eden od igralcev bankrotira. Naj bo $T$ število iger, ki je potrebnih, da eden od igralcev bankrotira.

\begin{itemize}
    \item Predpostavi, da je $T < \infty$ s.g. (*ali znaš to utemeljiti?). Določi verjetnost, da bankrotira igralec $M$.
    \item Predpostavi, da je $E[T] < \infty$. Določi $E[T]$.
\end{itemize}
\end{flushleft}

\section[Rešitev]{Rešitev}
Predpostavi, da je $E[T] < \infty$. Določi $E[T]$.
\bigskip

Rešujemo enačbo oblike $E[T|M=x] = p\cdot E[T|M=x+1] + (1-p)\cdot E[T|M=x-1] + 1$, kar zaradi lažjega zapisa prevedemo na obliko $f(x) = p\cdot f(x+1) + (1-p) \cdot f(x-1) + 1 ~ \textrm{oziroma} $ $$p\cdot f(x+2) - f(x+1) + (1-p)\cdot f(x) = -1$$ kjer je $f(x)=E[T|M=x]$. Rešujemo torej nehomogeno rekurzivno enačbo, katero rešitev je vsota homogene in partikularne rešitve. 
\\
Rešimo najprej homogeni del. Rešujemo enačbo: $$p\cdot f(x+2) - f(x+1) + (1-p)\cdot f(x) = 0$$ s pomočjo karakterističnega polinoma dobimo enačbo: $$p\cdot  \lambda ^2 - \lambda + (1-p) = 0$$ $$\lambda _{1, 2}= \frac{1 \pm \sqrt{1 - 4p(1-p)}}{2p} = \frac{1 \pm (2p-1)}{2p}.$$
Dobimo $\lambda _{1} = 1$ in $\lambda _{2} = \frac{1-p}{p}$. Rešitev homogene enačbe je torej oblike $A\cdot 1^x + B(\frac{1-p}{p})^x$, oziroma: $$A+ B \bigg( \frac{1-p}{p} \bigg )^x$$

Ker imamo homogeno rešitev sestavljeno iz dveh delov, bomo partikularni del prav tako poiskali v dveh delih. Prvi del iščemo z nastavkom $f(x)= Cx1^x$ oziroma $f(x)= Cx$. Vstavimo v prvotno enačbo in dobimo: $$pC(x+1)-C(x+1)+(1-p)Cx=-1$$ $$Cpx+2Cp-Cx-C+Cx-Cpx=-1$$ $$C(2p-1)=-1$$ $$C=\frac{-1}{2p-1} = \frac{1}{1-2p}$$ Prvi del partikularne enačbe je: $$f(x)=\frac{x}{1-2p}$$

Drugi del iščemo z nastavkom $f(x) = D x (\frac{1-p}{p})^x$. Vstavimo v prvotno enačbo in dobimo:
$$pD(x+2)\bigg( \frac{1-p}{p} \bigg )^{(x+2)} - D(x+1)\bigg( \frac{1-p}{p} \bigg )^{(x+1 )}+(1-p)Dx\bigg( \frac{1-p}{p} \bigg )^x = -1$$ 
$$D \cdot \bigg( \frac{1-p}{p} \bigg )^x \cdot \bigg( px\bigg( \frac{1-p}{p} \bigg )^2 + 2p\bigg( \frac{1-p}{p} \bigg )^2 -x\bigg( \frac{1-p}{p} \bigg )-\bigg( \frac{1-p}{p} \bigg )+x-px \bigg )= -1$$ 
$$D = \frac{-1}{\big( \frac{1-p}{p} \big )^x \cdot \big( px\big( \frac{1-p}{p} \big )^2 + 2p\big( \frac{1-p}{p} \big )^2 -x\big( \frac{1-p}{p} \big )-\big( \frac{1-p}{p} \big )+x-px \big )}$$ 
\\
Drugi del partikularne enačbe je: $$f(x) = \frac{-x  (\frac{1-p}{p})^x}{\big( \frac{1-p}{p} \big )^x \cdot \big( px\big( \frac{1-p}{p} \big )^2 + 2p\big( \frac{1-p}{p} \big )^2 -x\big( \frac{1-p}{p} \big )-\big( \frac{1-p}{p} \big )+x-px \big )}$$ $$= \frac{x}{(x+1)\big( \frac{1-p}{p} \big )-\big( \frac{1-p}{p} \big )^2(px+2p)+px-1},$$\\
celotna partikularna rešitev pa je vstoa homogene in partikularne: $$f(x)=\frac{x}{1-2p}+\frac{x}{(x+1)\big( \frac{1-p}{p} \big )-\big( \frac{1-p}{p} \big )^2(px+2p)+px-1}.$$\\

Splošna rešitev prvotne enačbe je: $$A + B\bigg( \frac{1-p}{p} \bigg )^x+\frac{x}{1-2p}+\frac{x}{(x+1)\big( \frac{1-p}{p} \big )-\big( \frac{1-p}{p} \big )^2(px+2p)+px-1}$$

Upoštevamo lahko še robna pogoja:
\begin{enumerate}
\item Pričakovano število iger, če smo brez denarja je 0, saj se takrat igra konča. Pogoj zapišemo v obliki $E[T|M= 0] = f(0) = 0$. Vstavimo v splošno rešitev in dobimo:
$$A + B\bigg( \frac{1-p}{p} \bigg )^0 = 0, \quad A = -B$$
\item Pričakovano število iger, če imamo m+n denarja je 0, saj se takrat igra konča. Pogoj zapišemo v obliki $E[T|M= m+n] = f(m+n) = 0$. Vstavimo v splošno rešitev in dobimo:\\\\
$-B + B\bigg( \frac{1-p}{p} \bigg )^{m+n}+\frac{m+n}{1-2p}+\frac{m+n}{(m+n+1)\big( \frac{1-p}{p} \big )-\big( \frac{1-p}{p} \big )^2(p(m+n)+2p)+p(m+n)-1}=0$\\

$$B = \frac{\frac{m+n}{1-2p}+\frac{m+n}{(m+n+1)\big( \frac{1-p}{p} \big )-\big( \frac{1-p}{p} \big )^2(p(m+n)+2p)+p(m+n)-1}}{1-\big( \frac{1-p}{p} \big )^{m+n}}$$

$$A = \frac{\frac{m+n}{1-2p}+\frac{m+n}{(m+n+1)\big( \frac{1-p}{p} \big )-\big( \frac{1-p}{p} \big )^2(p(m+n)+2p)+p(m+n)-1}}{\big( \frac{1-p}{p} \big )^{m+n}-1}$$
\end{enumerate}

Pričakovano število iger, če imamo $x$ enot denarja je: $$E[ T | M = x]= \frac{\frac{m+n}{1-2p}+\frac{m+n}{(m+n+1)\big( \frac{1-p}{p} \big )-\big( \frac{1-p}{p} \big )^2(p(m+n)+2p)+p(m+n)-1}}{\big( \frac{1-p}{p} \big )^{m+n}-1}+$$

 $$+\frac{\frac{m+n}{1-2p}+\frac{m+n}{(m+n+1)\big( \frac{1-p}{p} \big )-\big( \frac{1-p}{p} \big )^2(p(m+n)+2p)+p(m+n)-1}}{1-\big( \frac{1-p}{p} \big )^{m+n}}\big( \frac{1-p}{p} \big )^x+$$ $$+\frac{x}{1-2p}+\frac{x}{(x+1)\big( \frac{1-p}{p} \big )-\big( \frac{1-p}{p} \big )^2(px+2p)+px-1}$$ kar lahko poenostavimo na
 Zanima nas $E[T|M=m]$, vstavimo $x=m$ v zgornjo enačbo in dobimo, da je pričakovano število iger, če začnemo z $m$ enotami denarja enako: $$E[T|M=m]=\frac{n \left(\left(\frac{1}{p}-1\right)^m-1\right)-m \left(\frac{1}{p}-1\right)^m
   \left(\left(\frac{1}{p}-1\right)^n-1\right)}{(2 p-1)
   \left(\left(\frac{1}{p}-1\right)^{m+n}-1\right)}$$
 \\
 Formula ne drži če je $p = \frac{1}{2}$, saj pride do deljenja z 0, zato moramo to izračunati posebej. Prav tako rešujemo nehomogeno rekurzivno enačbo, ki jo zapišemo kot: $$\frac{1}{2}f(x+2)-f(x+1)+\frac{1}{2}f(x)=-1$$

Rešimo najprej homogeni del. Rešujemo enačbo: $$\frac{1}{2}f(x+2)-f(x+1)+\frac{1}{2}f(x)=0$$ s pomočjo karakterističnega polinoma dobimo enačbo: $$\frac{1}{2}\lambda^2-\lambda+\frac{1}{2}=0$$ Rešitev je dvojna ničla $\lambda_{1, 2}= 1$. Homogena rešitev je oblike: $$(Ax+B)\cdot 1^x=(Ax+B)$$

Partikularno rešitev iščemo z nastavkom $f(x)=C\cdot x^2\cdot 1^x= Cx^2$. Vstavimo v prvotno enačbo in dobimo:
$$C(x+2)^2-2C(x+1)^2+Cx^2=-2$$
$$C(x^2+4x+4)-2C(x^2+2x+1)+Cx^2=-2$$
$$2C=-2, \quad C = -1 $$ Partikularna rešitev je $f(x)=-x^2$, splošna rešitev pa je: $$f(x)=Ax+B-x^2$$
Vstavimo robne pogoje:
\begin{enumerate}
\item $f(0)=0\Rightarrow B = 0$
\item $f(m+n)=0 \Rightarrow A = m+n$
\end{enumerate}
Pričakovano število iger, če imamo $x$ enot denarja in je verjetnost za zmago $p=\frac{1}{2}$ je: $$E[T|M=x]= (m+n)x-x^2$$
Zanima nas pričakovano število iger, če začnemo z $m$ enotami denarja. Odgovor je:$$E[T|M=m]=(m+n)m-m^2= n\cdot m$$


\end{document}