\documentclass[10pt]{beamer}
\usepackage[slovene]{babel}
\usepackage[utf8]{inputenc}
\usepackage[T1]{fontenc}
\usepackage{lmodern}
\usepackage{mathptmx}
\usepackage{helvet}
\usepackage{courier}

\usetheme{CambridgeUS}

\begin{document}

\title[Kockarjev propad ]{Kockarjev propad}
\subtitle{Gambler's ruin}
\author{Eva Deželak in Ines Šilc}
\institute [FMF]{ Fakulteta za matematiko in fiziko}

\begin{frame}
	\titlepage
\end {frame}
\section[Posebni primeri]{Posebni primeri}
\begin{frame}
\section[Vsota verjetnosti]{Vsota verjetnosti}
Vsota verjetnostii, da bankrotira igralec $M$, in da bankrotira igralec $N$:\\
\begin{equation*}
\begin{split}
&~\frac{(1-p)^{m+n}-p^m (1-p)^n}{(1-p)^{m+n}-p^{m+n}}+\frac{p^{m+n}-p^m
   (1-p)^n}{p^{m+n}-(1-p)^{m+n}} =\\
= &~\frac{(1-p)^{m+n}-p^m (1-p)^n}{(1-p)^{m+n}-p^{m+n}}-\frac{p^{m+n}-p^m
   (1-p)^n}{(1-p)^{m+n}-p^{m+n}} =\\
= &~\frac{(1-p)^{m+n}-p^m (1-p)^n-p^{m+n}+p^m
   (1-p)^n}{(1-p)^{m+n}-p^{m+n}} =\\
= &~\frac{(1-p)^{m+n}-p^{m+n}}{(1-p)^{m+n}-p^{m+n}} \\
= &~1
\end{split} 
\end{equation*}

\end{frame}
\section[Matematično upanje za $p = \frac{1}{2}$]{Matematično upanje za $p = \frac{1}{2}$}
\begin{frame}
\frametitle{Matematično upanje za $p = \frac{1}{2}$}
$$E[T|M=x] = \frac{1}{2}\cdot E[T|M=x+1] + \frac{1}{2}\cdot E[T|M=x-1] + 1,$$
\begin{equation}
\label{druga rekurzivna}
\frac{1}{2}f(x+2)-f(x+1)+\frac{1}{2}f(x)=-1
\end{equation}\\
Homogeni del:\\
$$\frac{1}{2}f(x+2)-f(x+1)+\frac{1}{2}f(x)=0$$\\
Karakteristični polinom:
\begin{equation}
\label{druga homogena}
\frac{1}{2}\lambda^2-\lambda+\frac{1}{2}=0
\end{equation} \\
Rešitev je dvojna ničla $\lambda_{1, 2}= 1$. Rešitev enačbe (\ref{druga rekurzivna}) je oblike:\\ $$(Ax+B)\cdot 1^x=Ax+B$$
\end{frame}

\begin{frame}
Nastavek za iskanje partikularne rešitve: $f(x)=C\cdot x^2\cdot 1^x= Cx^2$
 Vstavimo v enačbo (\ref{druga rekurzivna}) in dobimo:
\begin{equation*}
\begin{split}
 & ~~C(x+2)^2-2C(x+1)^2+Cx^2=-2\\
\Leftarrow & ~~C(x^2+4x+4)-2C(x^2+2x+1)+Cx^2=-2\\
\Leftarrow &  ~~C = -1 \\
\end{split}
\end{equation*}

Partikularna rešitev je $f(x)=-x^2$, splošna rešitev enačbe (\ref{druga rekurzivna}) pa je: $$f(x)=Ax+B-x^2$$

Vstavimo robne pogoje:
\begin{enumerate}
\item $f(0)=0\Rightarrow B = 0$
\item $f(m+n)=0 \Rightarrow A = m+n$
\end{enumerate}
Rešitev: $$E[T|M=x]= (m+n)x-x^2$$
Če začnemo z $m$ enotami denarja:$$E[T|M=m]=(m+n)m-m^2= n\cdot m$$
\end{frame}



























\end{document}